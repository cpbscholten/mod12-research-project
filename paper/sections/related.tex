Attacks to low cost routers  and the use of honeypots has been studied by the security research community. This section elaborates on some of this research and on the impact of this research in the field of cybersecurity.

M. Niemietz and J. Schwenk \cite{ROUTERSEC:RUB:2015} evaluated routers from ten different manufacturers and shows how all of these are vulnerable for cross-site scripting attacks, UI redressing and fingerprinting attacks. The researchers were able to circumvent the security of all of the investigated routers. The research discusses how these vulnerabilities can be exploited and provides countermeasures to make home routers more secure.

Many low-cost routers are used to interconnect networks with the BGP protocol. In this context, M. Mujtaba et al. \cite{BGPANALYSIS:EDCOUN:2011} analyzed the BGP protocol for vulnerabilities and compared three statistical anomaly detection systems for BGP attacks which could potentially be used to increase the defenses of low-cost routers running this protocol.

A. Ghourabi et al. also discusses the abuse of routing protocols on low-cost routers \cite{HONEYPOT:IEETR:2009}. The paper discusses attacks based on the RIP protocol. The author used a honeypot as an effective way to capture attacks targeted at the RIP protocol. This proceeding offers a great insight in how a honeypot router can be created and used to capture real hacker attacks. 

P. Sokol et al. discusses the ethical and legal perspectives of using honeypots for research and also discusses the issue of liability with honeypots \cite{HONEYPOTSLIABILITY:SPRINGER:2015}. This problem arises when honeypots are exploited by attackers and used to launch new attacks. The paper discusses the systems that need to be taken into account when designing a honeypot to minimize the risks. Similarly, C. Hecker et al. \cite{HONEYPOT:MARY:2006} argue for the use of dynamic honeypots instead of static or low interaction or high interaction honeypots.

Despite of the popularity of MikroTik devices, there is only a limited number of works aiming to investigate RouterOS. While no research has been done on using RouterOS as a honeypot device, some research has been done on monitoring attacks on MikroTik RouterOS. The article ``Live Forensics on RouterOS using API Services to Investigate Network Attacks \cite{ROUTEROSFORENSICS:IJCSIS:2017}" discusses using live forensics on RouterOS as a technique to capture network attacks. The article specifically mentions that only internal attacks were researched and research should be done on using live forensics to discover network attacks from external networks. This research was fairly limited and only included a proof of concept attack and did not involve the monitoring and characterization of captured attacks.

This research adds substantial information to the fields of cybersecurity, as analyzing attacks on low-cost routers can give new insights in the characteristics of attacks. A lot of research is available on the different types of vulnerabilities, with some research available about the subject of honeypots. All in all, very little research is available on the characteristics of attacks on low-cost routers. The only research discussing attacks on MikroTik routers \cite{ROUTEROSFORENSICS:IJCSIS:2017} only captured a proof of concept attack and focused on attacks from the internal network. Characterizing real world attacks against routers will give a better insight in the intents of hackers and can be used to provide new defend mechanisms against attacks on low-cost routers.

