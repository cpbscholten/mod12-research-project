% This is "bach-ref-2009.tex" Updated january 29th 2010.
% This file should be compiled with "sig-alternate-fixed.cls" January 2010.
% It is based on the ACM style "sig-alternate.cls"
% -----------------------------------------------------------------------
% This example file demonstrates the use of the 'sig-alternate-fixed.cls'
% V2.5 LaTeX2e document class file. It is for those submitting
% articles to the Twente Student Conference on IT. Both this file as the 
% document class file are based upon ACM documents.
%
% ------------------------------------------------------------------------
% This .tex file (and associated .cls) produces:
%       1) The Permission Statement
%       2) The Conference (location) Info information
%       3) The Copyright Line TSConIT
%       4) NO headers and/or footers
%
%
% Using 'sig-alternate.cls' you have control, however, from within
% the source .tex file, over both the CopyrightYear
% (defaulted to 200X) and the ACM Copyright Data
% (defaulted to X-XXXXX-XX-X/XX/XX).
% e.g.
% \CopyrightYear{2007} will cause 2007 to appear in the copyright line.
% \crdata{0-12345-67-8/90/12} will cause 0-12345-67-8/90/12 to appear in the copyright line.
%
% ------------------------------------------------------------------------
% This .tex source is an example which *does* use
% the .bib file (from which the .bbl file % is produced).
% REMEMBER HOWEVER: After having produced the .bbl file,
% and prior to final submission, you *NEED* to 'insert'
% your .bbl file into your source .tex file so as to provide
% ONE 'self-contained' source file.
%

% refers to the cls file being used
\documentclass{sig-alternate-br}
\usepackage[hyphens]{url}
\usepackage[table]{xcolor}% http://ctan.org/pkg/xcolor
\usepackage{flushend}

\begin{document}
%
% --- Author Metadata here --- DO NOT REMOVE OR CHANGE 
\conferenceinfo{31$^{st}$ Twente Student Conference on IT}{Jul. 5$^{th}$, 2019, Enschede, The Netherlands.}
\CopyrightYear{2019} % Allows default copyright year (200X) to be over-ridden - IF NEED BE.
%\crdata{0-12345-67-8/90/01}  % Allows default copyright data (0-89791-88-6/97/05) to be over-ridden - IF NEED BE.
% --- End of Author Metadata ---

\title{Hacking the router: characterizing real attacks targeting low cost routers using a honeypot router}
% In Bachelor Referaat at University of Twente the use of a subtitle is discouraged.
% \subtitle{[Instructions]}

%
% You need the command \numberofauthors to handle the 'placement
% and alignment' of the authors beneath the title.
%
% For aesthetic reasons, we recommend 'three authors at a time'
% i.e. three 'name/affiliation blocks' be placed beneath the title.
%
% NOTE: You are NOT restricted in how many 'rows' of
% "name/affiliations" may appear. We just ask that you restrict
% the number of 'columns' to three.
%
% Because of the available 'opening page real-estate'
% we ask you to refrain from putting more than six authors
% (two rows with three columns) beneath the article title.
% More than six makes the first-page appear very cluttered indeed.
%
% Use the \alignauthor commands to handle the names
% and affiliations for an 'aesthetic maximum' of six authors.
% Add names, affiliations, addresses for
% the seventh etc. author(s) as the argument for the
% \additionalauthors command.
% These 'additional authors' will be output/set for you
% without further effort on your part as the last section in
% the body of your article BEFORE References or any Appendices.

\numberofauthors{3} %  in this sample file, there are a *total*
% of EIGHT authors. SIX appear on the 'first-page' (for formatting
% reasons) and the remaining two appear in the \additionalauthors section.
%
\author{
% You can go ahead and credit any number of authors here,
% e.g. one 'row of three' or two rows (consisting of one row of three
% and a second row of one, two or three).
%
% The command \alignauthor (no curly braces needed) should
% precede each author name, affiliation/snail-mail address and
% e-mail address. Additionally, tag each line of
% affiliation/address with \affaddr, and tag the
% e-mail address with \email.
%
% 1st. author
\alignauthor
Christian Scholten\\
       \affaddr{University of Twente}\\
       \affaddr{P.O. Box 217, 7500AE Enschede}\\
       \affaddr{The Netherlands}\\
       \email{c.p.b.scholten@student.utwente.nl}
}
% There's nothing stopping you putting the seventh, eighth, etc.
% author on the opening page (as the 'third row') but we ask,
% for aesthetic reasons that you place these 'additional authors'
% in the \additional authors block, viz.
% \additionalauthors{Additional authors: John Smith (The
% Th{\o}rv{\"a}ld Group, email: {\texttt{jsmith@affiliation.org}})
% and Julius P.~Kumquat (The Kumquat Consortium, email:
% {\texttt{jpkumquat@consortium.net}}).}
% \date{30 July 1999}
% Just remember to make sure that the TOTAL number of authors
% is the number that will appear on the first page PLUS the
% number that will appear in the \additionalauthors section.

\maketitle
\begin{abstract}
In this paper real hacker attacks against low cost routers will be investigated with the goal to get a better understanding of the intents of hackers and to find ways to improve defense mechanisms for these devices. This will be achieved by characterizing and analyzing real hacker attacks performed against a honeypot router device in a cloud environment. RouterOS from MikroTik will be recreated and run in a cloud environment to capture real hacker attacks. Furthermore, the different types of attacks against low cost routers will be discussed including which of these attacks can be mapped and how these attacks can be mapped.
\end{abstract}

% A category with the (minimum) three required fields (NOT USED in Bachelor Referaat)
% \category{H.4}{Information Systems Applications}{Miscellaneous}
%A category including the fourth, optional field follows...
% \category{D.2.8}{Software Engineering}{Metrics}[complexity
% measures, performance measures]

\keywords{low cost routers, MikroTik, RouterOS, honeypot, security, vulnerabilities, hacker attacks}

\section{Introduction}
Low cost routers have been a popular infrastructure device in underdevelopment countries, where they are used for expanding internet coverage in remote places. Low cost routers are cheap router devices which can be used as a home router or as a local area router with more advanced routing features such as the Border Gateway Protocol (BGP). These devices have been a popular target hackers with these attacks becoming more and more popular, with lots of regular news coverage of these devices actively being exploited. For example, the FBI discovered that hundreds of thousands of home routers were vulnerable against attacks from Russian hackers \cite{FBIRus:REUTERS:2018}.

There are many different vendors providing these low cost router devices. Some of the popular vendors include Huawei, TP-Link, NetGear and MikroTik. With over 1.6 million devices publicly visible \cite{MIKROTIK:SHODAN:2019}, MikroTik is a popular manufacturer of these low cost routers. In recent years, multiple vulnerabilities for MikroTik routers with a CVSS \cite{CVSS} score of 7 or higher were discovered \cite{CVELIST}. These vulnerabilities have been a source for many attacks, one of which included two hundred thousand compromised MikroTik routers being used for mining cryptocurrency \cite{MikroTikCryptoHack:PCMAG:2018}. To protect systems and improve the security of the internet, it is important to characterize the attacks to such devices. By doing that, we can understand the intends of the hackers and set proper defenses. 

In this paper, attacks on low cost routers will be characterized by using a honeypot router device in a cloud environment to capture real hacker attacks. A honeypot is a computer system intended to mimic targets of cyber attacks. These honeypots can be used to detect attacks or to deflect attacks from a real target \cite{HONEYPOTDEF:NORTON}. The honeypot for this research will mimic RouterOS from MikroTik. MikroTik RouterOS is an important subject to study, due to the number of vulnerabilities published in recent years and the number of devices in underdevelopment countries. Another reason to choose MikroTik is that MikroTik supports downloading current and previous RouterOS releases as an image to run in a virtual machine or cloud environment. This makes it possible to test with different versions of RouterOS without the need to buy a MikroTik device. To be able to successfully detect hacker attacks against the honeypot, research will be done on the different attacks and how these attacks can be mapped.
\section{Related works}
This section elaborates on related work to this research, after which the impact of this research is discussed.

M. Niemietz and J. Schwenk \cite{ROUTERSEC:RUB:2015} evaluated routers from ten different manufacturers and shows how all of these are vulnerable for XSS attacks, UI redressing and fingerprinting attacks. The researchers were able to circumvent the security of all of the investigated routers. It discusses how these vulnerabilities can be exploited and provides countermeasures to make home routers more secure.
 
One of the protocols vulnerable for attacks is the BGP protocol. In the memo of S. Murphy \cite{BGPHIJACKING:INTSOC:2006} the vulnerabilities in the BGP protocol are analyzed. The memo discusses how the BGP protocol on routers can be used to delete, forge, or reply data with the potential to disrupt network routing.

Honeypots have been a common tool used by researchers to detect hacker attacks and the conference proceeding "Honeypot router device for router protocols protection \cite{HONEYPOT:IEETR:2009}" uses a honeypot to capture hacker attacks on routers. The honeypot was used to capture a real RIP attack. The proceeding offers a great insight in how a honeypot router can be created and used to capture real hacker attacks. 

A similar conference from 2006 had a  proceeding about dynamic honeypots by C. Hecker, K. L. Nance and B. Hay \cite{HONEYPOT:MARY:2006}. This proceeding argues for the use of a dynamic honeypot instead of a static or low/high interaction honeypots and explains the ways to set up a dynamic honeypot router with honeyd.

While no research has been done on using RouterOS as a honeypot device, some research has been done on monitoring attacks on MikroTik RouterOS. The article "Live Forensics on RouterOS using API Services to Investigate Network Attacks \cite{ROUTEROSFORENSICS:IJCSIS:2017}" discusses using live forensics on RouterOS as a technique to capture hacker attacks. The article specifically mentions that only internal attacks were researched and research should be done on using live forensics to discover hacker attacks from external networks. This research was fairly limited and only included a proof of concept attack and did not involve any monitoring and characterization of real hacker attacks.

\subsection{Impact}
This research adds substantial information to the fields of analyzing attacks on low cost routers. A lot of research is available on the different types of vulnerabilities, with some research available about the subject of honeypots. All in all, very little research is available regarding the characteristics of real hacker attacks. The only research discussing attacks on MikroTik routers \cite{ROUTEROSFORENSICS:IJCSIS:2017} only captured a proof of concept attack and only focused on attacks from the internal network. Characterizing the real attacks against routers will provide a better insight in the intents of hackers and can be used to provide new defend mechanisms against attacks on low cost routers.

\subsection{Research aims}
In this research the following research questions will be answered. The research questions are ordered in such a way that the that all previous questions need to be answered before the next question can be answered.

\textbf{RQ1} What are the different types of attacks low cost routers are vulnerable to.\\
\textbf{RQ2} Which attacks on low cost routers can be mapped and what methodology could be used to map each type of attack?\\
\textbf{RQ3} How can attacks on low cost routers be characterized by analyzing real hacker attacks on a honeypot router?



\section{Methodology}
In the following section, the proposed method to answer each research questions will be discussed.

\subsection{On answering RQ1}
To answer \textbf{RQ1}, the different types of attacks that can be performed against low cost routers needs to be studied with the likelihood of these attacks happening. This will be done by with a literature review on the vulnerabilities in low cost routers and the different attacks that hackers perform on low cost router devices.

\subsection{On answering RQ2}
A understanding of which attacks on low cost routers can be mapped and the methodology to map these attacks is necessary to create the honeypot device. When knowing how these attacks can be mapped, choices can be made on which functionalities of the router should be recreated for the honeypot and how the attacks on the honeypot can be monitored. This is necessary to answer \textbf{RQ2} and ensure that the honeypot system will be most effective in capturing the attacks and allows for capturing the most important real hacker attacks on the device. To do this, testing will be done on a virtual machine running RouterOS, which will be used for testing purposes to recreate some of the possible attacks. The critical vulnerabilities in RouterOS will be tested with the intent to discover if these vulnerabilities can be exploited in order to gain access to the management interface or a root shell on the router.

\subsection{On answering RQ3}
To characterize the attacks on low cost routers in \textbf{RQ3}, the user interface or command line interface of a low cost router will be recreated and installed on a public IP-address as a honeypot device. The system will be set up to appear exactly as a regular router and will capture all requests and user interactions with the system. To appear as a convincing MikroTik router, it needs to be created such that the device appears convincingly enough as a MikroTik device with a vulnerable RouterOS installation with the same port setup visible on \url{shodan.io}. A visible Apache server, for example, could be an indicator to hackers that this is not a real routing device and make this research less effective. The captured attacks against the honeypot by real hackers will be analyzed to discover the intent of the hackers. This will be done by gathering statistics about the attacks performed on the honeypot by using the methodologies to map the different types of attacks from \textbf{RQ2} and it will involve analyzing the different types of attacks performed on the honeypot to characterize the different types of attacks and to discover the intent of the hackers. A literature review will be done and the statistics from the collected attacks on the honeypot router will be used in order to perform this characterization.
% placed above here as an ugly hack to get table to correct place...
\begin{table*}[t]
\centering \caption{Research Planning}
\begin{tabular}{|l|c|c|c|c|c|c|c|c|c|c|c|} \hline
Week&17&18&19&20&21&22&23&24&25&26&27\\ \hline 
Finding related research&\cellcolor{gray}&\cellcolor{gray}&&&&&&&&&\\ \hline
Draft proposal&&\cellcolor{gray}5/5&&&&&&&&&\\ \hline
Peer review draft proposal&&&\cellcolor{gray}7/5&&&&&&&& \\ \hline
Proposal&&&\cellcolor{gray}12/5&&&&&&&&\\ \hline
Researching attacks \textbf{RQ1}&&&\cellcolor{gray}&\cellcolor{gray}&&&&&&&\\ \hline
Mapping attacks \textbf{RQ2}&&&&\cellcolor{gray}&\cellcolor{gray}&&&&&&\\ \hline
Building honeypot&&&&&\cellcolor{gray}&\cellcolor{gray}&&&&&\\ \hline
Gathering attack data&&&&&&\cellcolor{gray}&\cellcolor{gray}&\cellcolor{gray}&&&\\ \hline
Analyzing attacks \textbf{RQ3}&&&&&&\cellcolor{gray}&\cellcolor{gray}&\cellcolor{gray}&&&\\ \hline
Draft paper&&&&&&&\cellcolor{gray}&\cellcolor{gray}&\cellcolor{gray}23/6&&\\ \hline
Peer review draft paper&&&&&&&&&&\cellcolor{gray}25/6&\\ \hline
Draft presentation&&&&&&&&&\cellcolor{gray}&\cellcolor{gray}&\cellcolor{gray}5/7\\ \hline
Final paper&&&&&&&&&&\cellcolor{gray}30/7&\\ \hline
Presentation&&&&&&&&&&&\cellcolor{gray}5/7\\ \hline
\end{tabular}
\end{table*}
\section{Preliminary findings}
In the preliminary research, information was searched for vulnerabilities in MikroTik RouterOS, to get a better understanding of the severity of the vulnerabilities and on how these vulnerabilities can be exploited. More research still needs to be done on the reproducability of these vulnerabilities and how these vulnerabilities can be detected. This is not a full list of vulnerabilities on RouterOS, but it does include some recent vulnerabilities with a high score on the CVSS scale.

\subsection{Vulnerabilities in RouterOS}
CVE-2018-7445 is a vulnerability with the maximum score of 10 following the CVSSv2 scale. The vulnerability involves a bug in the Server Message Block (SMB) service of RouterOS, which could cause a stack overflow. The overflow happens when before authentication, causing an unauthenticated hacker to be able to execute malicious code. According to Core Security, the exploit takes place in a function parsing NetBIOS names, receiving two stack allocated buffers. These buffers will be copied over to the destination buffer without any size validation on the original buffer. If the original buffers are larger than the destination buffer, a stack overflow will happen. With an appropriate payload, a hacker can use this to execute malicious code or even gain root \cite{CVE-2018-7445:CORESEC:2018}. The vulnerability has been solved by MikroTik in RouterOS 6.41.3/6.42rc27. A workaround is disabling the SMB functionality on the router \cite{CVE-2018-7445:CORESEC:2018}.

CVE-2018-1156 is a vulnerability in the license upgrade system of MikroTik routers, which can be used to trigger a buffer overflow on the router and allow a hacker to remotely execute code. The vulnerability could be exploited by a remotely authenticated user with the following request \cite{CVE2018-1156:Tenable:2018}:
\begin{verbatim}
    GET /ssl_conn.php?usrname=%s&passwd=%s&softid
    =%s&level=%d&pay_type=%d&board=%d HTTP/1.0
\end{verbatim}

CVE-2018-14847 was originally published as a low priority bug where an attacker could only gain read only access to all files on the router. The bug is a directory traversal error in WinBox, an application from MikroTik to remotely access to router. A researcher from Tenable Research discovered that there was more to this vulnerability than expected. The vulnerability could also be used to write files on the file-system via WinBox or HTTP, without requiring any authentication \cite{CVE-2018-14847:TENABLE:2018}. This discovery increased the CVSSv2 score from a 5 to the maximum score of 10. Tenable estimates that of the vulnerable devices, 70\% will remain vulnerable \cite{CVE-2018-14847:TENABLE:2018}.

CVE-2019-3934 is a similar vulnerability to CVE-2018-14847. This is another directory traversal vulnerability in the WinBox software, which allows the hacker read and write access to all files on the router. The main difference with CVE-2018-14847 is that this vulnerability requires authentication before it can be invoked  \cite{CVE-2019-3943:TENABLE:2019}.
\section{Planned results}
To answer the research questions, the research will deliver the following results:
\begin{enumerate}
    \item An overview of the different types of attacks low cost routers are vulnerable to.
    \item An overview of which of the attacks can be mapped and for each of these attacks a methodology for mapping them.
    \item An evaluation of real hacker attacks performed on the honeypot device and a characterization of these attacks to discover the intent of hackers and proper defense mechanisms against attacks on low cost routers.
\end{enumerate}
\section{Research planning}
The planning for the research project is summarized in Table 1.% \section{Conclusions}

% make columns equal length
\flushcolsend

%ACKNOWLEDGMENTS are optional
% \section{Acknowledgments}

% The following two commands are all you need in the
% initial runs of your .tex file to
% produce the bibliography for the citations in your paper.
\bibliographystyle{ieeetr}
\bibliography{sigproc}  % sigproc.bib is the name of the Bibliography in this case
% You must have a proper ".bib" file
%  and remember to run:
% latex bibtex latex latex
% to resolve all references
%
% ACM needs 'a single self-contained file'!
%
\vspace{50 mm}
\newpage
%APPENDICES are optional
% \appendix
% Appendix A
% \section{Headings in Appendices}
% The rules about hierarchical headings discussed above for the body
% of the article are different in the appendices. In the
% \textbf{appendix} environment, the command \textbf{section} is
% used to indicate the start of each Appendix, with alphabetic order
% designation (i.e. the first is A, the second B, etc.) and a title
% (if you include one).  So, if you need hierarchical structure
% \textit{within} an Appendix, start with \textbf{subsection} as the
% highest level. Here is an outline of the body of this document in
% Appendix-appropriate form:
% \subsection{Introduction}
% \subsection{The Body of the Paper}
% \subsubsection{Type Changes and  Special Characters}
% \subsubsection{Math Equations}
% \paragraph{Inline (In-text) Equations}
% \paragraph{Display Equations}
% \subsubsection{Citations}
% \subsubsection{Tables}
% \subsubsection{Figures}
% \subsubsection{Theorem-like Constructs}
% \subsubsection*{A Caveat for the \TeX\ Expert}
% \subsection{Conclusions}
% \subsection{Acknowledgments}
% \subsection{Additional Authors}
% This section is inserted by \LaTeX; you do not insert it. You just
% add the names and information in the
% \texttt{{\char'134}additionalauthors} command at the start of the
% document.
% \subsection{References}
% Generated by bibtex from your ~.bib file.  Run latex, then bibtex,
% then latex twice (to resolve references) to create the ~.bbl file.
% Insert that ~.bbl file into the .tex source file and comment out
% the command \texttt{{\char'134}thebibliography}.
% % This next section command marks the start of
% % Appendix B, and does not continue the present hierarchy
% \section{More Help for the Hardy}
% The sig-alternate.cls file itself is chock-full of succinct and
% helpful comments.  If you consider yourself a moderately
% experienced to expert user of \LaTeX, you may find reading it
% useful but please remember not to change it.
%\balancecolumns % GM June 2007
% That's all folks!

% \balancecolumns
\end{document}