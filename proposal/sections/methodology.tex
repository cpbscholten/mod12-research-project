In the following section, the proposed method to answer each research questions will be discussed.

\subsection{On answering RQ1}
To answer \textbf{RQ1}, the different types of attacks that can be performed against low cost routers needs to be studied with the likelihood of these attacks happening. This will be done by with a literature review on the vulnerabilities in low cost routers and the different attacks that hackers perform on low cost router devices.

\subsection{On answering RQ2}
A understanding of which attacks on low cost routers can be mapped and the methodology to map these attacks is necessary to create the honeypot device. When knowing how these attacks can be mapped, choices can be made on which functionalities of the router should be recreated for the honeypot and how the attacks on the honeypot can be monitored. This is necessary to answer \textbf{RQ2} and ensure that the honeypot system will be most effective in capturing the attacks and allows for capturing the most important real hacker attacks on the device. To do this, testing will be done on a virtual machine running RouterOS, which will be used for testing purposes to recreate some of the possible attacks. The critical vulnerabilities in RouterOS will be tested with the intent to discover if these vulnerabilities can be exploited in order to gain access to the management interface or a root shell on the router.

\subsection{On answering RQ3}
To characterize the attacks on low cost routers in \textbf{RQ3}, the user interface or command line interface of a low cost router will be recreated and installed on a public IP-address as a honeypot device. The system will be set up to appear exactly as a regular router and will capture all requests and user interactions with the system. To appear as a convincing MikroTik router, it needs to be created such that the device appears convincingly enough as a MikroTik device with a vulnerable RouterOS installation with the same port setup visible on \url{shodan.io}. A visible Apache server, for example, could be an indicator to hackers that this is not a real routing device and make this research less effective. The captured attacks against the honeypot by real hackers will be analyzed to discover the intent of the hackers. This will be done by gathering statistics about the attacks performed on the honeypot by using the methodologies to map the different types of attacks from \textbf{RQ2} and it will involve analyzing the different types of attacks performed on the honeypot to characterize the different types of attacks and to discover the intent of the hackers. A literature review will be done and the statistics from the collected attacks on the honeypot router will be used in order to perform this characterization.