Low cost routers have been a popular infrastructure device in underdevelopment countries, where they are used for expanding internet coverage in remote places. Low cost routers are cheap router devices which can be used as a home router or as a local area router with more advanced routing features such as the Border Gateway Protocol (BGP). These devices have been a popular target hackers with these attacks becoming more and more popular, with lots of regular news coverage of these devices actively being exploited. For example, the FBI discovered that hundreds of thousands of home routers were vulnerable against attacks from Russian hackers \cite{FBIRus:REUTERS:2018}.

There are many different vendors providing these low cost router devices. Some of the popular vendors include Huawei, TP-Link, NetGear and MikroTik. With over 1.6 million devices publicly visible \cite{MIKROTIK:SHODAN:2019}, MikroTik is a popular manufacturer of these low cost routers. In recent years, multiple vulnerabilities for MikroTik routers with a CVSS \cite{CVSS} score of 7 or higher were discovered \cite{CVELIST}. These vulnerabilities have been a source for many attacks, one of which included two hundred thousand compromised MikroTik routers being used for mining cryptocurrency \cite{MikroTikCryptoHack:PCMAG:2018}. To protect systems and improve the security of the internet, it is important to characterize the attacks to such devices. By doing that, we can understand the intends of the hackers and set proper defenses. 

In this paper, attacks on low cost routers will be characterized by using a honeypot router device in a cloud environment to capture real hacker attacks. A honeypot is a computer system intended to mimic targets of cyber attacks. These honeypots can be used to detect attacks or to deflect attacks from a real target \cite{HONEYPOTDEF:NORTON}. The honeypot for this research will mimic RouterOS from MikroTik. MikroTik RouterOS is an important subject to study, due to the number of vulnerabilities published in recent years and the number of devices in underdevelopment countries. Another reason to choose MikroTik is that MikroTik supports downloading current and previous RouterOS releases as an image to run in a virtual machine or cloud environment. This makes it possible to test with different versions of RouterOS without the need to buy a MikroTik device. To be able to successfully detect hacker attacks against the honeypot, research will be done on the different attacks and how these attacks can be mapped.